\documentclass[11pt]{article}
\usepackage{hyperref}

% ---------- Packages ----------
\usepackage{amsmath,amssymb,amsfonts,amsthm}
\usepackage{enumitem}
\usepackage{geometry}
\usepackage{fancyhdr}
\usepackage{hyperref}
\usepackage{xcolor}   
\usepackage{amsmath} 
\usepackage{amssymb}
\usepackage{amsfonts}
\usepackage{tikz}
\usepackage{xcolor}
\usepackage{environ} 

% ---------- Page Geometry ----------
\geometry{margin=1in}

% ---------- Running Header / Footer ----------
\pagestyle{fancy}
\fancyhf{}
\lhead{\textbf{CS 189/289A}}
\rhead{Fall 2025}
\cfoot{\thepage}

% ---------- Helpful Macros ----------
\newcommand{\RR}{\mathbb{R}}
\newcommand{\EE}{\mathbb{E}}
\newcommand{\PP}{\mathbb{P}}
\renewcommand{\vec}[1]{\mathbf{#1}}

\begin{document}

% ---------- Title Block ----------
\begin{center}
    {\huge\bfseries Homework 1}\\[4pt]
    \textbf{Due : Friday, September 19 at 11 : 59 pm}\\
\end{center}

\vspace{0.5em}\hrule\vspace{1em}

% ---------- Deliverables ----------
\noindent\textbf{Deliverables.} Submit a single PDF of your write-up to Gradescope \emph{HW1 Write-Up}. 
\\
\\
\noindent\textbf{Start each problem on a new page.}


\vspace{0.5em}\hrule\vspace{0.75em}


%%%%%%%%%%%%%%%%%%%%%%%%%%%%%%%%%%%%%%%%%%%%%%%%%%%%%%%%%%%%%%%%%%%%%%%%%%%%%%%%
\section*{Honor Code}

\noindent\emph{Write and sign the following statement:}

\begin{quote}\small
“I certify that all solutions in this document are entirely my own and that I have
not looked at anyone else’s solution. I have given credit to all external
sources I consulted.”
\end{quote}

%%%%%%%%%%%%%%%%%%%%%%%%%%%%%%%%%%%%%%%%%%%%%%%%%%%%%%%%%%%%%%%%%%%%%%%%%%%%%%%%
%%%%%%%%%%%%%%%%%%%%%%%%%%%%%%%%%%%%%%%%%%%%%%%%%%%%%%%%%%%%%%%%%%%%%%%%%%%%%%%%
%%%%%%%%%%%%%%%%%%%%%%%%%%%%%%%%%%%%%%%%%%%%%%%%%%%%%%%%%%%%%%%%%%%%%%%%%%%%%%%%
\section*{Linear Algebra}

\begin{enumerate}[label=\arabic*., resume]

    \item \textbf{System of equations (3 points).}
          Consider the linear system
          \[
             \begin{bmatrix}
                 1 & 2 & 3\\
                 2 & 0 & 6\\
                 3 & 6 & 9
             \end{bmatrix}
             \begin{bmatrix}x\\ y\\ z\end{bmatrix}
             \;=\;
             \begin{bmatrix} 6 \\ 8 \\ 12 \end{bmatrix}.
          \]
          How many solutions does this system have?  Explain your reasoning.

          % YOUR ANSWER HERE
          \newpage

    \item \textbf{Asymptotic powers of $2\times2$ matrices (6 points).}
For each matrix below, determine the behavior of 
\[
    \lim_{n\to\infty} M^{\,n}
\]

\textit{Hint: Use the eigen-decomposition $M = P D P^{-1}$, where $D = \{\lambda_1, \lambda_2, ...\}$ is a diagonal matrix of the eigenvalues. What would be the formula for $M^n$?}

\begin{enumerate}[label=(\alph*)]
    \item
    $\displaystyle
       M_1 \;=\;
       \begin{bmatrix}
           0 & 1\\
           1 & 0
       \end{bmatrix}.
     $
    % YOUR ANSWER HERE

    \item
    $\displaystyle
       M_2 \;=\;
       \begin{bmatrix}
           2 & -5\\
           1/2 & -7/6
       \end{bmatrix}.
     $
    % YOUR ANSWER HERE
    

    \item
    $\displaystyle
       M_3 \;=\;
       \begin{bmatrix}
           4 & -2\\
           1 & 1
       \end{bmatrix}.
     $
    % YOUR ANSWER HERE
    
\end{enumerate}
\newpage


    \item \textbf{Singular Value Decomposition (4 points).} Given the matrix
    
    $$A = 
    \begin{bmatrix}
      2 & -1 \\
      2 & 2
    \end{bmatrix}$$
        
    Find a unit vector $x$ ($\|x\|=1$) for which $\|Ax\|$ is maximized.
    
    \textit{Hint: Recall from \href{https://en.wikipedia.org/wiki/Singular_value_decomposition}{SVD} that $A = U \Sigma V^\top$, where $A$ maps each right singular vector (a column of $V$) to a left singular vector (a column of $U$), scaled by the corresponding singular value. For which right singular vector does this scaling make $\|Ax\|$ the largest?}

    % YOUR ANSWER HERE
    

\newpage
    \item\textbf{Image Flipping (6 points).}
        \textit{This question is a warmup to problem 4 in the coding section.} 
        Consider a tiny $3\times 3$ image represented by matrix $I$ and the same image flipped on the Y axis $I_{\text{flip}}$:
        \[
        I =
        \begin{bmatrix}
        x_1 & x_2 & x_3 \\
        x_4 & x_5 & x_6 \\
        x_7 & x_8 & x_9
        \end{bmatrix},
        \quad
        I_{\text{flip}} =
        \begin{bmatrix}
        x_7 & x_8 & x_9 \\
        x_4 & x_5 & x_6 \\
        x_1 & x_2 & x_3
        \end{bmatrix}.
        \]
        
        \begin{enumerate}[label=(\alph*)]
            \item What is the size of the transformation matrix $T$ that performs this flip? 
            \textit{Hint: You can first convert $I$ to a $1\times 9$ vector, make transformation using matrix $T$, and then convert $I_{\text{flip}}$ back to a $3\times 3$ matrix.}

            % YOUR ANSWER HERE
            
            \item Construct $T$ and verify that $I\times T$ produces $I_{\text{flip}}$.

            % YOUR ANSWER HERE
            
            \item Describe an algorithm for constructing the vertical-flip transformation matrix for any $N\times N$ matrix (either text explanation or pseudocode is acceptable). Show that $I\times T$ produces $I_{\text{flip}}$.

            % YOUR ANSWER HERE
            
            \item Describe how you would modify this algorithm to do a \textit{horizontal} flip (along the X ais).

            % YOUR ANSWER HERE
            
        \end{enumerate}
    
    
\end{enumerate}
\newpage




%%%%%%%%%%%%%%%%%%%%%%%%%%%%%%%%%%%%%%%%%%%%%%%%%%%%%%%%%%%%%%%%%%%%%%%%%%%%%%%%
\section*{Calculus}

\begin{enumerate}[label=\arabic*., resume]

    \item \textbf{Partial Derivatives (8 points). (First and second derivatives)}
          \begin{enumerate}[label=(\alph*)]
              \item For \(f(x_1, x_2) = x_1^3 + x_2^3 - 3x_1x_2\), find all the first and second order partial derivatives.

              % YOUR ANSWER HERE
              
              \item Let \( f(x, y) = 4x^2 + y^2 - 8xy + 4x + 6y - 10 \). 
                
                Find the critical points by solving 
                      \[
                          \frac{\partial f}{\partial x} = 0 \quad \text{and} \quad \frac{\partial f}{\partial y} = 0
                      \]
                      simultaneously and determine which point(s) yield a minimum value.

                % YOUR ANSWER HERE

            
              \item For \( f(x, y) = e^{xy} + x^2y \), compute:
                    \begin{enumerate}[label=(\roman*)]
                        \item \(\frac{\partial f}{\partial x}\) and \(\frac{\partial f}{\partial y}\)

                        % YOUR ANSWER HERE
                        
                        \item \(\frac{\partial^2 f}{\partial x^2}\), \(\frac{\partial^2 f}{\partial y^2}\), and \(\frac{\partial^2 f}{\partial x \partial y}\)

                        % YOUR ANSWER HERE
                        
                    \end{enumerate}
              \item For \( f(x, y) = \ln(x^2 + y^2 + 1) \), compute:
                    \begin{enumerate}[label=(\roman*)]
                        \item \(\frac{\partial f}{\partial x}\) and \(\frac{\partial f}{\partial y}\)

                        % YOUR ANSWER HERE
                        
                        \item \(\frac{\partial^2 f}{\partial x^2}\), \(\frac{\partial^2 f}{\partial y^2}\), and \(\frac{\partial^2 f}{\partial x \partial y}\)

                        % YOUR ANSWER HERE
                        
                    \end{enumerate}
          \end{enumerate}

       
    \newpage
    
        

    \item \textbf{Recursive expression and derivatives (6 points).}
          Suppose we have variables \(z_1, z_2, \dots, z_n\), \(w_1, \dots, w_{n-1}\), and \(b_1, \dots, b_{n-1} \in \mathbb{R}\), where
          \[
              z_n = w_{n-1}z_{n-1} + b_{n-1}.
          \]
          \begin{enumerate}[label=(\alph*)]
              \item Compute \(\dfrac{dz_k}{dz_{k-1}}\) for \(k = 2, \dots, n\).

              % YOUR ANSWER HERE
              
              \item Compute \(\dfrac{dz_n}{dz_1}\).

              % YOUR ANSWER HERE

              \item Compute \(\dfrac{dz_n}{db_1}\).
           You may express your answers in terms of $z_1,...,z_n,w_1,...,w_{n-1},b_1,...,b_{n-1}$

           % YOUR ANSWER HERE
           
          \end{enumerate}

          

\end{enumerate}
\newpage



%%%%%%%%%%%%%%%%%%%%%%%%%%%%%%%%%%%%%%%%%%%%%%%%%%%%%%%%%%%%%%%%%%%%%%%%%%%%%%%%
\section*{Probability}

\begin{enumerate}[label=\arabic*., resume]

    \item \textbf{Conditioned uniform difference (3 points).}
          Let $X,Y\stackrel{\text{iid}}{\sim}\text{Unif}(-1,1)$. Compute
          \[
             \PP\!\bigl(|X-Y|\le 0.5 \,\bigl|\, X\,Y>0\bigr).
          \]

            \textit{Hint: Try drawing a 2D cartesian plane where the horizontal and vertical axes represent X and Y respectively and each range from -1 to 1. For which quadrants is it true that $XY > 0$? Within these quadrants, how can we visualize the region $|X-Y| < 0.5$?}

            % YOUR ANSWER HERE

        
    \newpage

    \item \textbf{Nearest–neighbor arc length (4 points).}
        Suppose you select $20$ i.i.d.\ points $X_1,\dots,X_{20}$ uniformly at random on the
        circumference of the unit circle. 

        \begin{enumerate}
            \item Let D be the shortest arc distance from $X_1$ to the nearest of the other 19 points. Calculate $\mathbb{P}(D > t)$ where $0 \leq t \leq \frac{1}{2}$.

            \textit{Hint: What does the event $\{D > t\}$ mean in terms of where the other 19 points can be located relative to $X_1$?}

            % YOUR ANSWER HERE
            

            \item Find the expected arc length (in degrees)
            between $X_1$ and the point nearest to it. 
            
            \textit{Hint: The} \href{https://en.wikipedia.org/wiki/Expected_value#Random_variables_with_density}{Wikipedia page on expected value of continuous variables} \textit{may be helpful here.}

            % YOUR ANSWER HERE


        \end{enumerate}
    \newpage


    \item \textbf{Cancer screening (3 points).}
          A medical test has sensitivity $90\%$ and false–positive rate $3\%$:
          \[
          \PP(T=1\mid C=1)=0.9,\qquad \PP(T=1\mid C=0)=0.03.
          \]
          Suppose the disease prevalence is very low, $\PP(C=1)=0.001$. Compute the posterior probability of disease given a positive test:
          \[
          \PP(C=1\mid T=1).
          \]

          % YOUR ANSWER HERE

    \newpage

    \item \textbf{Follower Counts (3 points).}
        Suppose $420$ people are sitting uniformly at random around a circle, each with a
        \emph{distinct} number of TikTok followers. What is the expected number of
        people whose follower count is higher than both their immediate neighbors?

        % YOUR ANSWER HERE

\end{enumerate}

\end{document}

